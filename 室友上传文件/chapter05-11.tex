\documentclass[12pt,fancyhdr,UTF8,openany]{ctexart}
\usepackage{geometry}

\usepackage{amsmath,cases,graphicx,lastpage,paralist,amssymb,bm,esvect,yhmath,float,amsthm,mathtools,savesym,txfonts}
\usepackage{amsfonts}
\begin{document}
	\title{第五章与第十一章习题解答}
	\author{bald theory}
	\date{}
	\maketitle
\section*{第五章习题}

\noindent	2、设$K$是度量空间,$E$是赋范空间,$(f_n)_{n\geqslant 1}$是一列从$K$到$E$的连续函数,\par 
$(1)$证明若$(f_n)_{n\geqslant1}$在一个点$x$处等度连续,则对任一收敛到$x$的点列$(x_n)_{x\geqslant 1}$,都有$(f_n(x)-f_n(x_n))_{n\geqslant 1}$收敛到$0$。\par 
$(2)$证明如果$(f_n(x))_{n\geqslant 1}$在$E$中收敛到$y$,那么对任一收敛到$x$的点列$(x_n)_{n\geqslant1}$,$(f_n(x_n))_{n\geqslant1}$也收敛到$y$。\par 
$(3)$取$f_n(x)=\sin nx$。证明$(f_n)_{n\geqslant 1}$在$\mathbb{R}$上每一点都不等度连续。\par 
	\textbf{证明:}\par 
	$(1)$若$(f_n)_{n\geqslant 1}$在点$x$处等度连续,则对$\forall \varepsilon>0$,$\exists \delta>0$,使得当$d(x,y)<\delta$时,对$\forall n\in\mathbb{N}^*$,$||f_n(x)-f_n(y)||<\varepsilon$。且对任意收敛到$x$的点列$(x_n)_{n\geqslant 1}$,存在正整数$N$,当$n>N$时,$d(x_n,x)<\delta$,故此时对任意正整数$k$,$||f_k(x)-f_k(x_n)||<\varepsilon$。取$k=n$,得$||f_n(x)-f_n(x_n)||<\varepsilon$。\par 综上,对$\forall \varepsilon>0$,$\exists$正整数$N$,当$n>N$时,$||f_n(x)-f_n(x_n)||<\varepsilon$,这就是说$(f_n(x)-f_n(x_n))_{n\geqslant 1}$收敛到$0$。\par 
	$(2)$由于$(f_n(x))_{n\geqslant 1}$在$E$中收敛到$y$,故对$\varepsilon>0$,$\exists N_1\in\mathbb{N}^*$,当$n>N_1$时,$||f_n(x)-y||<\frac{\varepsilon}{2}$,又根据$(1)$,设$(x_n)_{n\geqslant 1}$收敛于$x$,则存在$N_2\in\mathbb{N}^*$,使得$n>N_2$时,$||f_n(x)-f_n(x_n)||<\frac{\varepsilon}{2}$。取$N=\max\left\{N_1,N_2\right\}$,当$n>N$时,
	\[||f_n(x_n)-y||\leqslant ||f_n(x_n)-f_n(x)||+||f_n(x)-y||\leqslant \dfrac{\varepsilon}{2}+\dfrac{\varepsilon}{2}<\varepsilon.\]\par 
	故$f_n(x_n)_{n\geqslant 1}$也收敛到$y$。\par 
	$(3)$ 若$x=k\pi$,$k\in\mathbb{Z}$,则取$x_n=k\pi+\frac{1}{n}$,注意到$f_n(k\pi)=\sin(nk\pi)=0$,故$\lim\limits_{n\rightarrow \infty}f_n(k\pi)=0$,而\[\begin{aligned}
		f_n(x_n)&=\sin(n(k\pi+\frac{1}{n}))\\&=\sin(nk\pi+1)\\
		&=\sin(nk\pi)\cos1+\cos(nk\pi)\sin1\\&=\cos(nk\pi)\sin1
	\end{aligned}\]
从而$|f_n(x_n)|=\sin 1$对任意正整数$n$都成立,因此$f_n(x_n)$在$n$趋于$\infty$时极限不可能为$0$。由$(2)$知$(f_n)_{n\geqslant 1}$在$x=k\pi$处不等度连续。\par 

若$x\not= k\pi$,$k\in\mathbb{Z}$,取$x_n=x+\frac{\pi}{n}$,从而
\[\begin{aligned}
||f_n(x)-f_n(x_n)||&=|\sin (nx)-\sin(nx+\pi)|\\
 &=|\sin (nx)-\sin(nx)\cos\pi-\cos(nx)\sin\pi|\\
                   &=2|\sin nx|
\end{aligned}\]\par 
下面我们说明当$x\not=k\pi$时,$\lim\limits_{n\rightarrow \infty}\sin nx$不存在。\par 
事实上,设$x\not=k\pi$,$k\in\mathbb{Z}$,若$\lim\limits_{n\rightarrow \infty} \sin nx$存在,那么
\[\lim\limits_{n\rightarrow \infty} (\sin((n+1)x)-\sin((n-1)x))=0.\]
由和差化积,我们知道$\sin((n+1)x)-\sin((n-1)x)=2\sin x\cos nx$,从而
\[\lim\limits_{n\rightarrow \infty} \cos nx=0.\]
接着注意到$\cos((n+1)x)=\cos nx\cos x-\sin nx\sin x$,故\[\lim\limits_{n\rightarrow \infty} \sin nx =0,\]而这与$\sin^2 nx+\cos^2 nx=1$矛盾!从而$\lim\limits_{n\rightarrow \infty} \sin nx$不存在。\par 
因此当$n$趋近于$\infty$时,$||f_n(x)-f_n(x_n)||$极限不存在,由$(1)$知$(f_n)_{n\geqslant 1}$在$x\not= k\pi$处不等度连续。\par 综上,$(f_n)_{n\geqslant 1}$在$\mathbb{R}$上每一点都不等度连续。$\square$\par 
\newpage
\noindent 5、考虑函数序列$(f_n)$,这里$f_n(t)=\sin(\sqrt{t+4(n\pi)^2}),t\in[0,\infty)$。\par 
$(a)$证明$(f_n)$等度连续并且逐点收敛到$0$函数。\par 
$(b)$ $C_b([0,\infty),\mathbb{R})$表示$[0,\infty)$上所有有界连续实函数构成的空间,并赋予范数
\[ ||f||_\infty=\sup_{t\geqslant 0}|f(t)|.\]\par 
$(f_n)$在$C_b([0,\infty),\mathbb{R})$中是否相对紧?\par 
\textbf{证明:}\par 
$(a)$ 对任意$t\geqslant 0$,
\[\begin{aligned}
	|f_n(t)|&=|\sin(\sqrt{t+4(n\pi)^2})|\\
	        &=|\sin(\sqrt{t+4(n\pi)^2}-2n\pi)|\\
	        &=|\sin(\dfrac{t}{\sqrt{t+4(n\pi)^2}+2n\pi})|\\
	        &\leqslant \dfrac{t}{\sqrt{t+4(n\pi)^2}+2n\pi}\rightarrow 0,(n\rightarrow \infty)
\end{aligned}\]\par 
故$(f_n)$逐点收敛到0函数。\par 
设$x\in[0,\infty)$,则对任意$\varepsilon>0$,取$\delta=2\pi\varepsilon$,当$|x-y|<\delta$时,
\[\begin{aligned}
	|f_n(y)-f_n(x)|&=|\sin\sqrt{y+4(n\pi)^2}-\sin\sqrt{x+(4(n\pi)^2)}\\
	               &=|2\cos(\dfrac{\sqrt{y+4(n\pi)^2}+\sqrt{x+(4(n\pi)^2)}}{2})\sin(\dfrac{\sqrt{y+4(n\pi)^2}-\sqrt{x+(4(n\pi)^2)}}{2})|\\
	               &\leqslant\dfrac{|y-x|}{\sqrt{y+4(n\pi)^2}+\sqrt{x+(4(n\pi)^2)}}\\
                   &\leqslant \dfrac{|x-y|}{2\pi}<\varepsilon.	              
\end{aligned}\]
从而$(f_n)_{n\geqslant 1}$等度连续。\par 
$(b)$ 注意到依范数$||\cdot||_\infty$下的收敛即为在$[0,\infty)$下的一致收敛,假设$(f_n)_{n\geqslant1}$有依范数$||\cdot||_\infty$收敛的子列,则由$(a)$知该子列必一致收敛于0函数,但是对任意正整数$n$,$||f_n||_\infty=\sup\limits_{t\geqslant0}|\sin\sqrt{t+4(n\pi)^2}|=1$,因此$(f_n)$的任一子列不可能收敛于0函数,导出矛盾,故$(f_n)$不是相对紧的。\newpage
10、设$(K,d)$是紧度量空间。证明所有从$K$到$\mathbb{R}$的Lipschitz函数构成的集合在$(C(K,\mathbb{R}),||\cdot||_{\infty})$中稠密。\par 
\textbf{证明:}记所有从$K$到$\mathbb{R}$的Lipschitz函数的集合为$Lip(K,\mathbb{R})$。容易验证$Lip(K,\mathbb{R})$是$C(K,\mathbb{R})$的向量子空间,下面证明它还是一个代数:\par 
设$f,g\in Lip(K,\mathbb{R})$,故存在$\lambda_1>0,\lambda_2>0$,使得对任意$x,y\in K$有
\[ |f(x)-f(y)|\leqslant \lambda_1 d(x,y),\\
|g(x)-g(y)|\leqslant \lambda_2 d(x,y). \]且注意到$K$是紧的,$f,g$都是连续的,从而存在$M_1>0,M_2>0$,使得对任意$x\in K$,$|f(x)|\leqslant M_1$,$|g(x)|\leqslant M_2$。从而对任意$x,y\in K$,我们有
\[\begin{aligned} 
	|f(x)g(x)-f(y)g(y)|&=|f(x)g(x)-f(y)g(x)+f(y)g(x)-f(y)g(y)|\\
	                   &\leqslant M_2 |f(x)-f(y)|+M_1 |g(x)-g(y)|\\
	                   &\leqslant (\lambda_1 M_2+\lambda_2 M_1)d(x,y)
\end{aligned}\]
因此$fg\in Lip(K,\mathbb{R})$。故$Lip(K,\mathbb{R})$是$C(K,\mathbb{R})$的子代数。\par 
对任意$x,y\in K$,且$x\not= y$,则取$f(x)=d(x,y)$,则$f(x)\not= 0$但$f(y)=d(y,y)=0$。从而$f(x)\not= f(y)$。因此$Lip(K,\mathbb{R})$在$K$上是可分点的。另一方面对任意$x\in K$,取$f(x)=1\in Lip(K,\mathbb{R})$,则$f(x)\not=0$。故由Stone-Weierstrass定理知,所有从$K$到$\mathbb{R}$的Lipschitz函数构成的集合在$(C(K,\mathbb{R}),||\cdot||_{\infty})$中稠密。\par
\vspace{2em}
12、$(1)$ $\left[0,1\right]$上所有的偶多项式构成的集合$\mathcal{Q}$是否在$C(\left[0,1\right],\mathbb{R} )$上稠密?\par 
$(2)$ $\left[-1,1\right]$上所有的偶多项式构成的集合$\mathcal{R}$是否在$C(\left[0,1\right],\mathbb{R})$上稠密?\par 
\textbf{解:}$(1)$ 容易验证$[0,1]$上所有偶多项式的集合$\mathcal{Q}$是一个代数。并且取$f(x)=x^2$,则对任意$x,y\in[0,1],x\not= y$,我们有$f(x)\not=f(y)$。并且对任意$x\in[0,1]$,取$f(y)=y^2+1,$从而$f(x)=x^2+1\not=0$。故由Stone-Weierstrass定理,$\mathcal{Q}$在$C([0,1],\mathbb{R})$上稠密。\par 
$(2)$设$f(x)=x$,若存在一列偶多项式$p_n(x)$,使得
\[\lim\limits_{n\rightarrow \infty} \sup_{x\in[-1,1]}|p_n(x)-f(x)|=0.\]
故对任意$x\in [-1,1]$,
\[\lim\limits_{n\rightarrow \infty}p_n(x)=f(x).\]
从而
\[f(-x)=\lim\limits_{n\rightarrow \infty}p_n(-x)=\lim\limits_{n\rightarrow \infty}p_n(x)=f(x).\]
这与$f(x)=x$是奇函数矛盾!故$\left[-1,1\right]$上所有的偶多项式构成的集合$\mathcal{R}$不在$C(\left[0,1\right],\mathbb{R})$上稠密。
\section*{第十一章习题}
8、设$E$和$F$是赋范空间。证明下面的命题成立:\par 
$(a)$ 若$(x_n)$是$E$的弱收敛序列,则$(x_n)$有界。\par 
$(b)$ 若$T\in\mathcal{B}(E,F)$且$x_n$弱收敛到$x$,则$T(x_n)$弱收敛到$T(x)$。\par 
$(c)$ 若$T\in\mathcal{B}(E,F)$是紧算子且$x_n$弱收敛到$x$,则$T(x_n)$依范数收敛到$T(x)$。\par 
$(d)$ 若$E$自反,$T\in\mathcal{B}(E,F)$且当$x_n$弱收敛到$x$时,有$T(x_n)$依范数收敛到\par $T(x)$,则$T$是紧算子。\par 
$(e)$ 若$E$自反,且$T\in\mathcal{B}(E,l_1)$或$T\in\mathcal{B}(c_0,E)$,则$T$是紧算子。\par 
\textbf{证明:}$(a)$若$(x_n)$是$E$中的弱收敛序列,设其极限为$x$,则对任意$f\in E^*$,有$\lim\limits_{n\rightarrow \infty}f(x_n)=f(x)$。令$\hat{x_n}(f)=f(x_n)$,则$\hat{x_n}\in\mathcal{B}(E^*,\mathbb{R})$,且对任意$f\in E^*$,$(\hat{x_n}(f))_{n\geqslant 1}$有界。因此由Banach-Steinhaus定理,$\sup\limits_{n\geqslant 1}||x_n||=\sup\limits_{n\geqslant 1}||\hat{x_n}||<\infty$,从而$(x_n)_{n\geqslant 1}$有界。\par 
$(b)$ 若$T\in\mathcal{B}(E,F)$,则$T^*\in\mathcal{B}(F^*,E^*)$,则对任意$f\in F^*$,
\[ \lim\limits_{n\rightarrow \infty}<f,T(x_n)>=\lim\limits_{n\rightarrow \infty}<T^*(f),x_n>=<T^*(f),x>=<f,T(x)>.\]\par 
$(c)$ 由$(a)$知$(x_n)_{n\geqslant 1}$有界,而$T$是紧算子,故$(T(x_n))_{n\geqslant 1}$相对紧,从而$(T(x_n))_{n\geqslant 1}$任一子列必有收敛子列$(T(x_{n_k}))_{k\geqslant 1 }$,且该子列必依范数收敛于$T(x)$,这是因为若$(T(x_{n_k}))_{k\geqslant 1}$依范数收敛到$y$,则对任意$f\in F^*$,我们有
\[|f(T_{x_{n_k}})-f(y)|\leqslant ||f||~||T(x_{n_k})-y||.\]
因此$T_{x_{n_k}}$弱收敛于$T(y)$,但由$(b)$知,$T(x_n)$弱收敛于$T(x)$,因此$y=f(x)$。\par 若$T(x_n)$不依范数收敛于$T(x)$,则对任意正整数$N$,存在$n_0>N$,使得$||T(x_{n_0})-T(x)||>1$
,从而可选取一子列$(T(x_{n_k}))_{k\geqslant 1}$,使得$||T(x_{n_k})-T(x)||\geqslant1$,但由上述讨论可知$T((x_{n_k}))_{k\geqslant 1}$有收敛于$T(x)$的子列,矛盾!因此$T(x_n)$依范数收敛于$T(x)$。\par 
$(d)$由第九章第4题的结论,若$E$是的自反的,设$(x_n)_{n\geqslant 1}\subset B_E$,则存在子列$(x_{n_k})_{k\geqslant 1}$,使得$(x_{n_k})_{k\geqslant1}$弱收敛于$x$。由题目条件知$T(x_{n_k})$依范数收敛到$T(x)$,这说明$T(B_E)$相对紧,从而$T$是紧算子。
$(e)$由第八章第22题的结论知,$l_1$的依范数收敛与弱收敛等价。故若$l_1$中的序列$(x_n)$弱收敛到$x$,则$(x_n)$依范数收敛到$x$,从而$T(x_n)$依范数收敛到$T(x)$,由$(d)$知,$T$是紧算子。\par 
若$T\in \mathcal{B}(c_0,E)$,则$T^*\in \mathcal{B}(l_1,E^*)=\mathcal{B}(l_1,E)$,故由上一段讨论知$T^*$是紧算子,从而$T$是紧算子。\par 
\vspace{2em}
9、设$(e_n)$是$l_2$中的标准基。定义算子$T:l_2\rightarrow l_2$为
\[T(\sum_{n\geqslant 1}x_n e_n)=\sum_{n\geqslant 1}\dfrac{x_n}{n}e_n,\quad (x_n)_{n\geqslant 1}\in l_2.\]
证明:$T\in \mathcal{K}(l_2)$。\par 
\textbf{证明:}定义
\[T_N(\sum_{n\geqslant 1}x_n e_n)=\sum_{n=1}^{N}\dfrac{x_n}{n}e_n.\]
则$T_N$是有限秩算子,且
\[\begin{aligned}
	||T(\sum_{n\geqslant 1}x_n e_n)-T_N (\sum_{n\geqslant 1}x_n e_n)||^2&=||\sum_{n=N+1}^{\infty}\dfrac{x_n}{n}e_n||^2\\
	        &=\sum_{n=N+1}^\infty \dfrac{|x_n|^2}{n^2}\\
	        &\leqslant \sum_{n=N+1}^\infty \dfrac{1}{n^2}\cdot\sum_{n=N+1}^\infty |x_n|^2\\
	        &\leqslant \sum_{n=N+1}^\infty \dfrac{1}{n^2}\cdot \sum_{n=1}^\infty |x_n|^2.
\end{aligned}\] 
因此$\lim\limits_{N\rightarrow \infty}||T-T_N||\leqslant\lim\limits_{N\rightarrow \infty} \sqrt{\sum_{n=N+1}^\infty \frac{1}{n^2}}=0$,从而$T$是紧算子。\newpage
10、设$(\alpha_n)_{n\geqslant 1}\subset\mathbb{C}$。定义算子$T\in \mathcal{B}(c_0)$为
\[T(x)=(\alpha_n x_n)_{n\geqslant1},\quad x=(x_n)_{n\geqslant 1}\in c_0.\]
证明:$T\in \mathcal{K}(c_0)$当且仅当$\lim\limits_{n\rightarrow \infty}\alpha_n=0$。\par 
\textbf{证明:}必要性:若$\lim\limits_{n\rightarrow \infty} \alpha_n\not=0$,则存在子列$(\alpha_{n_k})_{k\geqslant 1}$,使得对任意$k\geqslant 1$,$|\alpha_{n_k}|\geqslant 1$。设$e_i=(0,\cdots,0,1,0,\cdots)$(在第$i$个坐标处是1,其余坐标都是0), 则$A=\left\{ e_i;i\geqslant 1\right\}$在$c_0$中有界,且
\[Te_{n_j}=\alpha_{n_j}e_{n_j},\]
但对任意$i\not= j$,
\[||Te_{n_j}-Te_{n_i}||=\max\left\{|\alpha_{n_i}|,|\alpha_{n_j}|\right\}>1,\]
故$(Te_{n_j})_{j\geqslant1}$的任一子列都不是Cauchy列,从而不收敛,这说明$T(A)$在$c_0$不是相对紧的。这与$T$是紧算子矛盾,故$\lim\limits_{n\rightarrow \infty}\alpha_n=0$。\par 
充分性:设$T_N(x)=(\alpha_n x_n)_{1\leqslant n\leqslant N}$,则$T_N$是有限秩算子。
且
\[\begin{aligned}
	||T(x)-T_N(x)||&=\sup_{n\geqslant N+1}|\alpha_n x_n|\\ 
	               &\leqslant \sup_{n\geqslant N+1}|\alpha_n| \sup_{n\geqslant 1}|x_n|.
\end{aligned}\]
由于$\lim\limits_{N\rightarrow \infty} \sup\limits_{n\geqslant N+1}|\alpha_n|=\limsup\limits_{n\rightarrow \infty}|\alpha_n|=\lim\limits_{n\rightarrow \infty}|\alpha_n|=0$。故
\[\lim\limits_{N\rightarrow \infty}||T-T_N||\leqslant  \lim\limits_{N\rightarrow \infty}\sup_{n\geqslant N+1}|\alpha_n|=0.\]
从而$T$是紧算子。

\end{document}