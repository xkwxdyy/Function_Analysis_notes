\chapter{拓扑向量空间}
    \section{拓扑向量空间}
    \begin{theorem}\label{thm:拓扑向量空间是正则空间}
        设$E$是一个拓扑向量空间, 则$E$是一个正则空间.
    \end{theorem}
    \begin{analysis}
        \begin{enumerate}
            \item 由$E$是一个拓扑向量空间可以将问题“平移”到0和不包含0的闭集$F$可分离
            \item $0\notin E \Rightarrow 0\in E^c$为开集, 那么就得到了0的一个开领域, 而对于0的开领域我们是知道一些性质, 进行运用即可证明.
        \end{enumerate}
    \end{analysis}
    \begin{proof}
        \begin{step}
            \item 先证明0与不包含0的闭集$F$可以用开领域分离
                \begin{step}
                    \item 由$0\notin F \Rightarrow 0\in F^c$为开集, 故令$V=F^c$为0的开领域, 由于$E$是一个拓扑向量空间, 故$\exists \tilde{U}\in \cN(0), \st \tilde{U}+\tilde{U}\subset V$, 而0有平衡的领域基, 故$\exists U$为0的平衡开领域, \st $U \subset \tilde{U}$, 因此$U+U \subset \tilde{U}+ \tilde{U} \subset V$, 而由$U$平衡知$U=-U$, 因此$U-U\subset V$.
                    \item 下证$U+F$是$F$的开领域. 由于$F\subset U+F(\forall y\in F \xLongrightarrow{0\in U} y=0+y)$, 且由$U$为开集知$U+F$为开集, 因此$U+F$是$F$的开领域.
                    \item 下证$U\cap (U+F)=\varnothing$. 若结论不成立, 则$\exists y\in U\cap (U+F)$, 故$\exists z\in U, x\in F \st y=z+x$, 故$x=y-z\in U-U \zj V=\{x\}^c$, 矛盾!
                \end{step}

                综上知0与$F$可被开集分离.
            \item 证明$\forall x\in E$, $F$为不包含$x$的一个闭集, 则$x$与$F$可以用开集分离, 即证明$E$是正则空间.
                \begin{step}
                    \item 由$x\notin F$知$0\notin F-x$, 否则$\exists y\in F \st 0=y-x, \ie x=y\in F$, 矛盾!
                    \item 由$F$为闭集及$\tau_{-x,1}$为闭映射知$F-x=\tau_{-x,1}(F)$为闭集, 因此由Step1及Step2.1知, 存在0的开领域$U$及$F-x$的开领域$V$, \st $U\cap V=\varnothing$.
                    \item 令$U_x=U+x, V_x=V+x$, 则$F=(F-x)+x\zj  V+x=V_x$, 而$x\in U_x$, 且由$U,V$为开集及平移算子$\tau_{x,1}$是开映射知$U_x=\tau_{x,1}(U), V_x=\tau_{x,1}(V)$分别为$x,F$的开领域.
                    \item 下证$U_x \cap V_x=\varnothing$, 否则$\exists y \in U_x \cap V_x= (U+x)\cap (V+x) \so \exists y_U \in U, y_V\in V \st y=y_U +x=y_V+x \so y_U=y_V\in U\cap V= \varnothing$, 矛盾!
                \end{step}
            
                综上所述, 命题得证. \qedhere
        \end{step}
    \end{proof}
    \section{半赋范空间}
    对于范数的相关性质,其实我们已经研究得比较好了,所以想要考虑一个范数的“减弱版本”——半范数,但是我们感兴趣的是\emphasize{非范数}的半范数,而由命题“$p$-拓扑是Hausdorff拓扑当且仅当$p$是一个范数”知道单单一个非范数的半范数诱导的拓扑并不是Hausdorff拓扑.

    但是我们想要研究问题往往希望拓扑至少有Hausdorff拓扑的性质,那怎么办呢?一个不行,那就搞\emphasize{一族}半范数,通过某种方式来诱导出\emphasize{一个}拓扑,这也是定义7.2.3的motivation.
    \section{局部凸空间}
    这一节的研究的整体目标其实很明确, 就是证明局部凸空间也是可半赋范的.

    关于这个研究的思路还是想说多一点, 这种方法概括起来其实是:
    \begin{step}
        \item 先研究命题$P$的必要条件$A$
        \item 然后考虑这个必要条件$A$是否也是$P$的充分条件
        \item 如果是的话, 那是再好不过的, 因为这相当于对$P$有了一个新角度的刻画
        \item 如果不是的话, 我们也会“挣扎一下”, 就会去想能不能在$A$的基础上加什么条件, 使得新的命题$A\pie$是$P$的充分条件.
    \end{step}

    但是从这一节的整体结果来看, 结论还是很漂亮的, 也就是我们得到的必要条件$A$(此处为“局部凸”)同时也是$P$(此处为“半赋范空间”)的一个充分条件.
    \begin{remark}
        不过有一点需要注意的是, 这两个等价都是在\emphasize{拓扑向量空间}的前提下考虑的.
    \end{remark}

    首先我们看到这一节一开始就给出的定义:
    \begin{definition}[定义7.3.1 局部凸空间]\label{def:局部凸空间}
        一个拓扑向量空间$E$称为局部凸空间,若它的原点有一组由凸集组成的领域基.
    \end{definition}
    这个定义是课本定理7.2.8中抽象出来的一个概念, 但是我们注意到, 它只对\emphasize{原点}有这个性质的叙述, 那么一个很自然的问题是, 局部凸空间的\emphasize{其它点}是否也有凸集组成的领域基呢? 不然为什么叫“局部凸”空间而不叫“原点凸”空间呢? 答案也符合我们的预期, 抽象出来就是下面的命题:
    \begin{proposition}\label{prop:任何一点有凸领域基}
        设$(E,\tau)$是一个局部凸空间,即它的原点有一组由凸集组成的邻域基,那么$E$中任何一点处也有由凸集组成的领域基.
    \end{proposition}
    在证明\autoref{prop:任何一点有凸领域基}之前我们需要证明另一个命题,就是凸性的“平移不变性”:
    \begin{proposition}[凸性的平移不变性质]\label{prop:凸性平移不变}
        设$E$是一个向量空间,$U\subset E$是一个凸集, 则$\forall x \in E$,$U+x$是$E$中的凸集.
    \end{proposition}
    \begin{proof}
        $\forall \lambda \in [0,1], \forall y_x,z_x \in U+x, \exists y,z \in U$, \st $y_x=y+x, z_x=z+x$.
        而由$U$为凸集知$\lambda y +(1-\lambda)z \in U$,因此
        \begin{align*}
            \lambda y_x +(1-\lambda) z_x &=\lambda(y+x)+(1-\lambda)(z+x)\\
            &=\big(\lambda y +(1-\lambda)z\big)+\bigl(\lambda+(1-\lambda)\bigr)x\\
            &=\big(\lambda y +(1-\lambda)z\big)+x\in U+x,
        \end{align*}
        故有$U+x$是凸集.
    \end{proof}

    \begin{remark}
        凸性的“平移不变性”也很符合我们的直觉,“凸性”只与\emphasize{集合本身}有关,于它“\emphasize{所处的位置}”无关.
    \end{remark}
    
    下面我们回到\autoref{prop:任何一点有凸领域基}的证明:
    \begin{proof}[\autoref{prop:任何一点有凸领域基}的证明]
        任取$x\in E ,\forall V \in \cN(x)$,由$E$为拓扑向量空间知$V-x\in \cN(0)$,而由$E$为局部凸空间知$\exists$凸集 $U$为$0$的开领域, \st  $U \subset V-x$. 而由$E$是一个拓扑向量空间知平移算子$\tau_{x,1}$是同胚映射,因而是一个开映射,所以$U+x=\tau_{x,1}(U)$是开集.

        又有\autoref{prop:凸性平移不变} 知$U+x$为凸集,故$U+x$为$x$的凸开领域,且满足$U+x \subset V$, 故$x$处也有凸集组成的领域基.
    \end{proof}
    
    课本在给出了局部凸空间的定义后给出的一个定理7.3.2也是这一节除了最终结论外我认为最重要的定理:
    \begin{theorem}[定理7.3.2]
        设$E$是一个局部凸空间, 则其在原点有凸的平衡开(或闭)邻域基.
    \end{theorem}

    这个定理的证明中运用到了凸包的概念, 这里作一个补充:
    \begin{definition}[凸包]\label{def:凸包}
        设$E$是一个向量空间, $B\zj E$为非空子集, 则包含$B$的最小凸集称为$B$的凸包, 记为$\operatorname{conv}B$.
    \end{definition}
    \begin{property}\label{property:凸包是凸集}
        凸包是凸集.
    \end{property}
    \begin{remark}\label{remark:凸包的等级定义:包含集合的最小凸集}
        和闭包的定义类似, 容易证明$B$的凸包的等价定义是包含$B$的所有凸集的交集.(可能读者会纳闷, 这本书的闭包定义并没有牵涉到这个, 但是在一些其它的拓扑教材中, 闭包被定义为包含集合的最小闭集或者是包含集合的所有闭集的交集, 和此处的凸包的定义方法完全相同).
    \end{remark}
    
    虽然我们在 \autoref{remark:凸包的等级定义:包含集合的最小凸集} 中提到了凸包的等价定义, 但是凸包的这两个定义都是“定性”的角度去看, 但是由于凸包是一个凸集(\autoref{property:凸包是凸集}), 而凸集又是一个“量化”的概念, 所以我们想, 凸包能否也有一个“定量”的等价刻画呢? 果然如同我们所预料的, 凸包可以由凸组合的形式“量化”, 也就是下面的命题:
    \begin{proposition}\label{prop:凸包的凸组合定义}
        \begin{equation}\label{eq:凸包的凸组合定义}
            \operatorname{conv}(B)=\set{\sum_{i=1}^{n} \lambda_{i} b_{i}}{ 0 \leqslant \lambda_{i} \leqslant 1, \sum_{i=1}^{n} \lambda_{i}=1, b_{i} \in B, 1 \leqslant i \leqslant n, n \geqslant 1}
        \end{equation}
    \end{proposition}

    其实我们可以直观理解一下这个命题为什么成立, 凸包是什么, 包含集合的最小凸集, 但是凸集是什么, 就是两点之间的点都在里面, 那么多个点的一种融洽组合(融洽指的是此处的线性组合前面的系数和为1)应该也在里面, 这是最起码要有的(最小性), 那么把这些点拿出来构成一个集合, 够了吗? 可以很容易验证 \autoref{eq:凸包的凸组合定义} 右边的集合是个凸集, 那这不就是最小的凸集吗? 那这应该就是所需要的凸包了.
    
    在证明 \autoref{prop:凸包的凸组合定义} 前我们需要一个下面的命题:
    \begin{proposition}\label{prop:凸组合包含于凸集中}
        设$B$是向量空间$E$的子集, 若$B$是凸集, 则
        \[\set{\sum_{i=1}^{n} \lambda_{i} b_{i}}{ 0 \leqslant \lambda_{i} \leqslant 1, \sum_{i=1}^{n} \lambda_{i}=1, b_{i} \in B, 1 \leqslant i \leqslant n, n \geqslant 1} \zj B\]
    \end{proposition}
    \begin{proof}
        数学归纳法.
    \end{proof}

    \begin{proof}[\autoref{prop:凸包的凸组合定义}的证明]
        两边相互包含, 简要证明如下:

        容易验证右边的集合是包含$B$的凸集, 由凸包的定义, 左包含于右.

        注意到凸包是凸集, $B\zj \operatorname{conv}B$以及 \autoref{prop:凸组合包含于凸集中} 可以证明右边包含于左边
    \end{proof}
    \begin{detail}{139}{\textbf{注7.3.3} 我们给出几个 Minkowski 泛函的性质:

        (1) 设
        \[
        I(x)=\left\{\lambda>0: \frac{x}{\lambda} \in \Omega\right\}
        \]
        若 $\lambda \in I(x)$ 且 $\mu>\lambda$, 则 $\mu \in I(x)$. 事实上, 由于 $\Omega$ 是平衡的, 则
        \[
        \frac{x}{\mu}=\frac{\lambda}{\mu} \frac{x}{\lambda} \in \Omega
        \]
        故 $I(x)$ 是以 $p_{\Omega}(x)$ 为左端点的半直线.}
        事实上这个性质不需要是平衡的, 只需要$\Omega$是包含原点的凸集即可:
        \[
        \frac{x}{\mu}=\frac{\lambda}{\mu} \frac{x}{\lambda} =\frac{\lambda}{\mu} \frac{x}{\lambda} + \left(1-\frac{\lambda}{\mu}\right)0 \in \Omega
        \]
        \begin{remark}
            之所以考虑到这个, 是因为在8.2节的定理8.2.1中也构造出了Minkowski泛函, 但是那个地方的$C$并没有“平衡”的条件, 而却依旧满足(iii):
            \[ \forall x \in E, p(x)<1 \Leftrightarrow x \in C.\]
            而这个结论是依赖于这个P139的注7.3.3性质(1)的, 所以就会去想这个为什么成立, 想到马老师当时上课的时候的证明就是用凸集来证明的, 当时还觉得老师的证明咋和书上不太一样, 但没有特别在意, 还好还有点印象.
        \end{remark}
        
    \end{detail}